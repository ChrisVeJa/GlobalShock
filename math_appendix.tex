\documentclass[12pt, a4paper]{article}
\renewcommand{\baselinestretch}{1.15}
\usepackage[left=3cm,right=3cm, top=3cm]{geometry}

\usepackage{enumerate}
\usepackage{amsmath}                                       % 
\usepackage{amsthm}                                        % 
\usepackage{amssymb}                                       % 
\usepackage[english]{babel}  % Caracteres en ESPAÑOL
\usepackage[utf8]{inputenc}
\usepackage{authblk}
%\usepackage{lastpage}
\usepackage{graphicx}
%\usepackage{authblk}
%\setlength{\parindent}{0pt}
\usepackage{fancyhdr}

\pagestyle{fancy}
\fancyhf{}
\fancyhead[CE,CO]{}
%\fancyfoot[RE,LO]{Banco Central de Reserva del Perú}
%\fancyfoot[LE,RO]{\thepage/\pageref{LastPage} }

\renewcommand{\headrulewidth}{0pt}
\renewcommand{\footrulewidth}{0pt}


\title{\textbf{Global optimism shocks for small open economies: Methodological appendix}}
\author{Velasquez, C.}
%\affil{Central Reserve Bank of Peru}
\date{}


\begin{document}
	
\maketitle

\section{The Uhlig's procedure}
In this section as an introduction to the global optimism identification we present the procedure used in Uhlig (2003) to identify anticipated shocks and extend it considering an exogenous block that is relevant to analyse small open economies.
 \subsection{The model}
 To make easier the algebra, we assume a VAR with N variables ( $n^*$ external variables, and $n$ domestic variables),an exogenous block, and 1 lag. Its reduce form is:
	\begin{equation}
	   \begin{aligned}
	     \left[ \begin{array}{c}  y^{(e)}_t  \\ y^{(d)}_t \end{array} \right] 
	    =
      	\left[ \begin{array}{cc}  F_{ee}^{(1)} & 0 \\ F_{de}^{(1)} & F_{dd}^{(1)} \end{array} \right] 
		\left[ \begin{array}{c}  y^{(e)}_{t-1}  \\ y^{(d)}_{t-1}  \end{array} \right] 
	 	+
    	\left[ \begin{array}{c}  u^e_{t}  \\ u^d_{t}  \end{array} \right] 
	  \end{aligned}
	\end{equation} \\
	The reduced errors, $u^e_{t}$ and $u^d_{t}$ are normally distributed with a variance-covariance matrix $\Sigma$. \\
	The respective \textit{orthogonalized} representation of the last system is:
	\begin{equation}
	   \begin{aligned}
	     \left[ \begin{array}{c}  y^{(e)}_t  \\ y^{(d)}_t \end{array} \right] 
	    & =
		\left[ \begin{array}{cc}  F_{ee}^{(1)} & 0 \\ F_{de}^{(1)} & F_{dd}^{(1)} \end{array} \right] 
		\left[ \begin{array}{c}  y^{(e)}_{t-1}  \\ y^{(d)}_{t-1}  \end{array} \right] 
	 	+
    	\left[ \begin{array}{cc}  C_{ee}^{(1)} & 0 \\ C_{de}^{(1)} & C_{dd}^{(1)} \end{array} \right]  *\left[ \begin{array}{c}  \epsilon^e_{t}  \\ \epsilon^d_{t}  \end{array} \right]  \\
	   Y_t & = F Y_{t-1} + C E_t
	\end{aligned}
	\end{equation} \\	
Both, $\epsilon^e_{t}$ and $\epsilon^d_{t}$ are \textit{orthogonal} errors distributed as a multivariate normal with mean 0 and a variance-covariance matrix I. Matrix C satisfies the restriction $CC' = \Sigma$, for convenience the matrix C could be obtained from a Cholesky decomposition. \\
One useful post-estimation tool of VAR models is the forecast error variance decomposition (FEVD), which is obtained following next step. The forecast error will be:
\begin{equation*}
\begin{aligned}
	Y_{t+h} & = F^h Y_t + \sum_{i=0}^h F^{i} C e_{t+h-i}\\
	Y_{t+h}  - E[T_{t+h}|t]& = \sum_{i=0}^h F^{i} C e_{t+h-i} \\
	 & = \sum_{i=0}^h R_{i} e_{t+h-i}
\end{aligned}
\end{equation*}
The matrix $R_i$ represents the impulse-response matrix $i$ periods ahead, with a Cholesky identification. Then, the forecast variance error is:
\begin{equation}
	 \Omega^{(h)} = \sum_{i=0}^h R_{i} R'_{i}
\end{equation}
The diagonal of $\Omega$ registers the forecast error variance $h-step$ ahead for each variable of the system.
\subsection{The Uhlig's identification scheme}
Th scheme of identification of Uhlig, also known as medium-run identification starts realizing that $E = D E^1$ while $DD' = I$. It means that any new identification matrix C*D will be observationally equivalent to the base model if D satisfies the aforementioned condition. It implies that:
\begin{equation*}
	 \Omega = \sum_{i=0}^h R_{i}D D' R'_{i}
\end{equation*}
A second fact is that $\Omega$ can be decomposed into the contribution of  its drivers as follow:
\begin{equation}{\label{eq:n2}}
	 \Omega = \sum_{i=0}^h \sum_{j=1}^{j=n} R_{i}\gamma_j \gamma_j' R'_{i}
\end{equation}
The equation \ref{eq:n2} decomposes that FEV in the contribution of each structural shock identified through $\gamma_j$ (the index represents the j-column in the matrix D). Commonly, this vector is called a vector identification. The original problem finds a identification vector $\gamma$ which maximizes the contribution of the j- structural shock in the accumulated forecast error variance (S) matrix between the periods $[\underline{\tau}, \overline{\tau}]$.\\
To solve the above describe problem, first we define $S$ as:
\begin{equation}
	 S(\underline{\tau},\overline{\tau}) =  \sum_{h = \underline{\tau}}^{\overline{\tau}}\sum_{i=0}^h \sum_{j=1}^{j=n} R_{i}\gamma_j \gamma_j' R'_{i}
\end{equation}
The contribution of a given structural shock $j$ identified by $\gamma_j$ is:
\begin{equation}
S(\underline{\tau},\overline{\tau}|j) =  \sum_{h = \underline{\tau}}^{\overline{\tau}}\sum_{i=0}^h R_{i}\gamma \gamma' R'_{i}
\end{equation}
The goal in Uhlig's problem is maximizing the contribution of a $j$ structural shock in the accumulated forecast variance of a external variable $k$, then we need to extract the $kxk$ position of $S(\underline{\tau},\overline{\tau}|j)$. We can write the problem as:
\begin{equation}\label{eq:n1}
\begin{aligned}
& \gamma^* &= \underset{\gamma}{\text{argmax}} 
& \left(\sum_{h = \underline{\tau}}^{\overline{\tau}}\sum_{i=0}^h R_{i}\gamma \gamma' R'_{i} \right)_{kk}\\
& \text{s.t.} 
&  & \gamma' \gamma = 1 \\
&  & & \gamma(z)  = 0 \quad \forall \ z>n^*
\end{aligned}
\end{equation}
The first condition allows to identify $\gamma$. The second restriction ensures that only external variables explain the forecast variance of $k$ ($n^*$ is the number of external variables in the base model). A discussion about this point will be develop at the end of this section. \\
A way to make easier (\ref{eq:n1}) is defining a matrix $G_{NxN}$ of zeros with 1 in the $kxk$ position, and using the trace operator and its properties. Then:
\begin{equation*}
\begin{aligned}
\left(\sum_{h = \underline{\tau}}^{\overline{\tau}}\sum_{i=0}^h R_{i}\gamma \gamma' R'_{i} \right)_{kk} & = \sum_{h = \underline{\tau}}^{\overline{\tau}}\sum_{i=0}^h tr(G R_{i}\gamma \gamma' R'_{i} G') \\
& = \sum_{h = \underline{\tau}}^{\overline{\tau}}\sum_{i=0}^h \underbrace{tr(\gamma' R'_{i} G'G R_{i}\gamma )}_{scalar} \\
& = \sum_{h = \underline{\tau}}^{\overline{\tau}}\sum_{i=0}^h \gamma' R'_{i} G'G R_{i}\gamma \\
& = \gamma' \left( \sum_{h =\underline{\tau}}^{\overline{\tau}}\sum_{i=0}^h  R'_{i} G'G R_{i} \right) \gamma \\
& = \gamma' \left( \sum_{h =\underline{\tau}}^{\overline{\tau}}\sum_{i=0}^h  (G R_{i})'(G R_{i}) \right) \gamma
\end{aligned}
\end{equation*}
It is easy to verify that $(G R_{i})'(G R_{i}) = R^{'(k)}_i*R^{(k)}_i$, where $R^{(k)}$ represents the k-row of matrix $R_i$. On the other hand. the above equation can be re-expressing as:
\begin{equation*}
\begin{aligned}
\gamma' \left( \sum_{h =\underline{\tau}}^{\overline{\tau}}\sum_{i=0}^h  (G R_{i})'(G R_{i}) \right) \gamma &= \gamma' \left( \sum_{h =\underline{\tau}}^{\overline{\tau}}\sum_{i=0}^h  R^{'(k)}_i R^{(k)}_i \right) \gamma \\ 
&= \gamma' \underbrace{\left(\sum_{i=0}^H (\overline{\tau}+1-max(\underline{\tau},i)) R^{'(k)}_iR^{(k)}_i \right)}_{\Lambda} \gamma
\end{aligned}
\end{equation*}
Finally, the Uhlig's problem is:
\begin{equation*}
\begin{aligned}
& \gamma^* &= \underset{\gamma}{\text{argmax}}  \quad
& \gamma' \Lambda  \gamma \\
& \text{s.t.} 
&  & \gamma' \gamma = 1 \\
&  & & \gamma(z)  = 0 \quad \forall \ z>n^*
\end{aligned}
\end{equation*}
After plant the Lagrangian it is easy to see that the solution is similar to a eigenvalue equation,  therefore we need to find the eigenvector related to the maximum eigenvalue of $\Lambda^*$ which is a submatrix of $\Lambda$ that includes the first $n*$ rows and columns. \\
It is easy to see that this expression is valid for both, a period of truncation or just a point of truncation. For the later, we have to define $\underline{\tau} = \overline{\tau}$ turning $\Lambda$ into:
\begin{equation*}
\Lambda = \sum_{i=0}^{\overline{\tau}} R^{'(k)}_iR^{(k)}_i 
\end{equation*}
Regarding the second restriction, we expand the expression $\gamma' \left( \sum_{h =\underline{\tau}}^{\overline{\tau}}\sum_{i=0}^h  R_{i}^{'(k)}R_{i}^{(k)} \right) \gamma$, without the sum operator, and considering the fact the each k-variable is an external variable:
\begin{equation*}
\begin{aligned}
R^{'(k)}_i\gamma' \gamma R_i^{(k)} &= \left[ \begin{array}{c}  C_{ee}^{'(i)}  \\ 0 \end{array} \right] \left[ \begin{array}{cc}  \gamma'_e & \gamma'_d  \end{array} \right] \left[ \begin{array}{c}  \gamma_e  \\ \gamma_d  \end{array} \right] 
\left[ \begin{array}{cc}  C_{ee}^{(i)} &  0 \end{array} \right] \\
 &= \left[ \begin{array}{cc} C^{'(i)}_{ee}\gamma'_e & C^{'(i)}_{ee}\gamma'_d  \\ 0 & 0 \end{array} \right]
 \left[ \begin{array}{cc} \gamma_e C^{(i)}_{ee} & 0  \\ \gamma_d C^{(i)}_{ee} & 0 \end{array} \right] \\
 &= \left[ \begin{array}{cc} C^{'(i)}_{ee}\gamma'_e \gamma_e C^{(i)}_{ee} + C^{'(i)}_{ee}\gamma'_d \gamma_d C^{(i)}_{ee}  & 0 \\ 0 & 0 \end{array} \right]
\end{aligned}
\end{equation*}
It is obvious that the previous matrix has a reduce rank ($n^*$), so we need to include $n$ additional restriction over $\gamma$. In this way, a valid restriction is consider $\gamma_d equal$ as a zero vector. Therefore, these restrictions allow us to achieve a exact identification, and the relevant matrix to solve the problem, will be the sub-matrix of the first $n^*$ rows and columns.
\clearpage  
\section{The Global Optimism methodology}
Since the work of Uhlig (2003) medium run identification has been used to identify effects of productivity and recently in anticipated shocks. This methodology finds an identification vector ($\gamma$) that allows to extract a structural shock with a maximum participation in fluctuations over a period $[\underline{\tau}:\overline{\tau}]$ of an objective variable. \\
We extend this methodology to obtain a vector $\gamma$ which maps structural shocks with maximal participation in N variables fluctuations (in this case, we consider external variable). \\
Defining $S^{i}(\underline{\tau},\overline{\tau}|j)$ the forecast variance of the i-variable explained by the $j$ structural shock over the span $[\underline{\tau}:\overline{\tau}]$ we can modify the maximization problem of Uhlig as: 
\begin{equation*}
\begin{aligned}
& \gamma^* &= \underset{\gamma}{\text{argmax}}  \quad
&  \sum_i^{n^*}\frac{S^{i}(\underline{\tau},\overline{\tau}|\gamma)}{S^{i}(\underline{\tau},\overline{\tau})} \\
& \text{s.t.} 
&  & \gamma' \gamma = 1 \\
&  & & \gamma(z)  = 0 \quad \forall \ z>n^*
\end{aligned}
\end{equation*}
Clearly, our objective is to find a identification vector with a maximal joint participation  in the forecast variance of all the $N^*$ external variables. It can be seem as recovering the underlying driver of the global economy, hence we call it as the global optimism strategy.\\
The first restriction $\gamma' \gamma = 1$ is an orthogonality condition to ensure that new identification will be observationally equivalent to the original estimation. The second restriction is made to identify the exogenous block. In contrast with Barsky \& Sims (2011) this condition does not involve a restriction about the contemporaneous effect of anticipated shock over real variables, instead it implies that external variable are the sources of their forecast error variance.\\
Rewriting the problem: 
\begin{equation*}
\begin{aligned}
\sum_i^{n^*}\frac{S^{i}(\underline{\tau},\overline{\tau}|\gamma)}{S^{i}(\underline{\tau},\overline{\tau})} 
 &=  \sum_i^{n^*}\frac{\gamma' \Lambda^{(i)} \gamma}{S^{i}(\underline{\tau},\overline{\tau})} 
\end{aligned}
\end{equation*}
Finding the common denominator:
\begin{equation*}
\begin{aligned}
\sum_i^{n^*}\frac{\gamma' \Lambda^{(i)} \gamma}{S^{i}(\underline{\tau},\overline{\tau})} 
&= 
\frac{\sum_i^{n^*} \left( \prod_{j=1,j\neq i}^{n^*} S^{i}(\underline{\tau},\overline{\tau})\right) \left( \gamma' \Lambda^{(i)} \gamma \right)}{\prod_i^{n^*} S^{i}(\underline{\tau},\overline{\tau})}
\end{aligned}
\end{equation*}
We realize that $\gamma$ are independent of the sum, and since the denominator is common we can retire it form the maximization.
\begin{equation*}
\begin{aligned}
\sum_i^{n^*}\frac{\gamma' \Lambda^{(i)} \gamma}{S^{i}(\underline{\tau},\overline{\tau})} 
& \propto 
\gamma' \left( \sum_i^{n^*} \left( \prod_{j=1,j\neq i}^{n^*}  S^{i}(\underline{\tau},\overline{\tau})\right) \Lambda^{(i)} \right) \gamma = \gamma' \xi  \gamma
\end{aligned}
\end{equation*}
Therefore, the problem is:
\begin{equation*}
\begin{aligned}
\gamma^* =  & \quad \underset{q_j}{\text{argmax}}
& & \gamma' \xi \gamma \\
& \text{s.t.}
& & \gamma' \gamma = 1\\
& & & \gamma(z) = 0 \ \forall \ z > n^*
\end{aligned}
\end{equation*}
Finally, $\gamma^*$ is the eigenvector related to the maximum eigenvalue of $\xi$.
\subsection{Identifying a non-fundamental shocks}
In order to identify a non-fundamental shock, which is defined as an innovation that is not related to global movements, we present a novel way to find a second anticipated shock orthogonal to the previously identified GOS. \\
We define $\Lambda^{(i)}$ as the accumulated FEV matrix of an $i$ variable. Then, the system to solve is:
\begin{equation*}
\begin{aligned}
q^*_k =  & \quad \underset{q_k}{\text{argmax}}
& & q'_k \Lambda^{(i)} q_k \\
& \text{s.t.}
& & q'_k q_k = 1\\
& & & q'_k \gamma = 0 \\
& & & q'_k(z) = 0 \ \forall \ z > n^*
\end{aligned}
\end{equation*}
The new restriction $q'_k \gamma = 0$ is imposed to ensure that the second identification vector will be orthogonal to the GOS. \\ 
The key to solve it is to realize that any linear combination of excluded eigenvectors of $\xi$ fulfils the orthogonality condition. So we can rewrite $q_k = Q^{(2)}a$, with $Q^{(2)}$ a matrix of size $n^*$ x $n^*-1$ composed of discarded eigenvectors of $\xi$ and $a$ a vector of weights. \\
Re-expressing the system:
\begin{equation*}
\begin{aligned}
q^*_k =  & \quad \underset{q_k}{\text{argmax}}
& & a' Q^{(2)'} \Lambda^{(i)} Q^{(2)} a \\
& \text{s.t.}
& & \sum a(i) = 1
\end{aligned}
\end{equation*}
Expanding $ a' Q^{(2)'} \Lambda^{(i)} Q^{(2)} a$ :
\begin{equation*}
\begin{aligned}
a' Q^{(2)'} \Lambda^{(i)} Q^{(2)} a &= a' \Psi a \\
&=  [a_1 \ a_2  \ a_3 \ ... \ a_{n^*-1}] 
\left[\begin{array}{ccc} \psi_{11} & \hdots & \psi_{1,n^*-1}  \\ \vdots & \ddots & \vdots \\ \psi_{n^*-1,1} & \hdots & \psi_{n^*-1,n^*-1} \end{array} \right]
\left[\begin{matrix} a_1 \\ a_2 \\ a_3 \\ \vdots \\ a_{n^*-1} \end{matrix} \right] \\
&=  a_1 \psi_{11} a_1 + a_2 \psi_{21} a_1 + + a_{(n^*-1)} \psi_{(n^*-1,1)} a_1+a_1 \psi_{21} a_1+ ... +a_1 \psi_{(n^*-1,1)} a_1 + ...
\end{aligned}
\end{equation*}
The respective Lagrangian is:
\begin{equation*}
\begin{aligned}
 \mathcal{L} = & a_1 \psi_{11} a_1 + a_2 \psi_{21} a_1 + + a_{(n^*-1)} \psi_{(n^*-1,1)} a_1+a_1 \psi_{21} a_1+ ... +a_1 \psi_{(n^*-1,1)} a_1 + ... \\ & + \lambda (1-a_1-a_2 - ...)
\end{aligned}
\end{equation*}
The first order conditions are:
\begin{equation*}
\begin{aligned}
\frac{\partial \mathcal{L}}{\partial a_1} &= a_1 (\psi_{11}+ \psi_{21} + ... + \psi_{(n^*-1,1)})+a_2 \psi_{21}+ ... + a_{(n-1)} \psi_{(n-1,1)}- \lambda=0 \\
\frac{\partial \mathcal{L}}{\partial a_2} &= a_2 (\psi_{12}+\psi_{22}+...+\psi_{(n-1,2))}+a_1 \psi_{12}+...+a_{(n-1)} \psi_{(n-1,2)}-\lambda=0 \\
\vdots \\
\frac{\partial \mathcal{L}}{\partial a_{n^*-1}} &= a_{(n-1)} (\psi_{(1,n-1 )}+\psi_{(2,n-1)}+ ... +\psi_{(n-1,n-1))}+a_1 \psi_{(1,n-1)}+...+a_{(n-2)} \psi_{(n-2,n-1)}-\lambda=0 \\
\frac{\partial \mathcal{L}}{\partial \lambda}   &=  1-a_1-a_2-...-a_{(n-1)}=0\\
\end{aligned}
\end{equation*}
An equivalent system is:
\begin{equation*}
\begin{aligned}
\frac{\partial \mathcal{L}}{\partial a_1} &= a_1 c_1+a_2 \psi_{21}+...+a_{(n-1)} \psi_{(n-1,1)}-\lambda=0 \\
\frac{\partial \mathcal{L}}{\partial a_2} &= a_1 \psi_{12} +a_2 c_2+ ...+a_{(n-1)} \psi_{(n-1,2)}-\lambda=0 \\
\vdots \\
\frac{\partial \mathcal{L}}{\partial a_{n^*-1}} &= a_1 \psi_{(1,n-1)}+a_2 \psi_{(2,n-1)}+...+a_{(n-1)} c_{(n-1)}-\lambda=0 \\
\frac{\partial \mathcal{L}}{\partial \lambda}   &=  1-a_1-a_2-…-a_{(n-1)}=0 \\
\end{aligned}
\end{equation*}
To solve it, we can subtract successively equation (i) - (i-1) forming:
\begin{equation*}
\begin{aligned}
a_1 (c_1-\psi_{12} )+a_2(\psi_{21}-c_2)+...+a_{(n-1)} (\psi_{(n-1,1)}-\psi_{(n-1,2)})=0 \\
a_1 (\psi_{12}-\psi_{13} )+a_2 (c_2-\psi_{23})+...+a_{(n-1)} (\psi_{(n-1,2)}-\psi_{(n-1,3)})=0 \\
\vdots \\
a_1 (\psi_{(1,n-2)}-\psi_{(1,n-1)} )+a_2 (\psi_{(2,n-2)}-\psi_{(2,n-1 )})+...+a_{(n-1)} (\psi_{(n-1,n-1)}-c_{(n-1)})=0 \\
a_1+a_2+...+a_{(n-1)}=1 
\end{aligned}
\end{equation*}
Or:
\begin{equation*}
\begin{aligned}
F * \left[\begin{matrix} a_1 \\ a_2 \\ a_3 \\ \vdots \\ a_{n^*-1} \end{matrix} \right] &=  \left[\begin{matrix} 0 \\ 0 \\ 0\\ \vdots \\ 1 \end{matrix} \right] \\
a &=  F^{-1} *\left[\begin{matrix} 0 \\ 0 \\ 0\\ \vdots \\ 1 \end{matrix} \right] 
\end{aligned}
\end{equation*}
Then,
\begin{equation*}
q_k = Q^{(2)}*a
\end{equation*}
And finally:
\begin{equation*}
q_k^* = \frac{q_k}{|| q_k ||}
\end{equation*}
Since we define an external block composed only for 3 variables, identification of a non-fundamental shock is easier,and based on a simple searching algorithm, following describe:
\begin{equation*}
\begin{aligned}
&a = [0.000 : 0.001: 1] \\
&s = nan(numel(a),1) \\
  & for \quad i  =  1:numel(a) \\
  & \quad h = a(i)* \lambda_2 + (1-a(2)*\lambda_3 \\
  & \quad h = h/sqrt(h'*h) \\
  & \quad s(i) = h'*\Lambda^{(ctot)}*h \\
  & end \\
&pond = a(find(s==max(s)))\\
&pond = pond(1); \\
&qk   = pond * \lambda_2 + (1-pond)* \lambda_3; 
\end{aligned}
\end{equation*}

\end{document}